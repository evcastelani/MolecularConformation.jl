
	\begin{table}[H]
        \centering
        {\footnotesize
        
        \begin{tabular}{||r |S[table-format=1.2e+2] |S[table-format=1.2e+2] |S[table-format=1.2e+2] |S[table-format=1.2e+2] |S[table-format=1.2e+2] |S[table-format=1.2e+2] | S[table-format=1.2e+2]||}
                \hline
				        $n_d$ & \multicolumn{1}{c|}{3} & \multicolumn{1}{c|}{4} & \multicolumn{1}{c|}{5} & \multicolumn{1}{c|}{10} & \multicolumn{1}{c|}{100} & \multicolumn{1}{c|}{200} & \multicolumn{1}{c||}{300} \\
        \hline
        Quater. & 6.76e-05 & 7.45e-05 & 7.79e-05 & 1.14e-04 & 1.01e-03 & 1.75e-03 & 2.14e-03 \\
        Matrix & 1.03e-04 & 1.11e-04 & 1.14e-04 & 1.53e-04 & 1.06e-03 & 1.82e-03 & 2.22e-03 \\ 
        $t_q\slash t_m$ & \multicolumn{1}{c|}{0.66} & \multicolumn{1}{c|}{0.67} & \multicolumn{1}{c|}{0.68} & \multicolumn{1}{c|}{0.75} & \multicolumn{1}{c|}{0.95} & \multicolumn{1}{c|}{0.96} & \multicolumn{1}{c||}{0.96} \\
        \hline
	\end{tabular}}
	\caption{Pratical improvement ratio of QuaternionBP in relation to MatrixBP considering the geometric means of benchmark times over all instances with fixed $n_d$.}
	\label{table:improvlavor}
\end{table}