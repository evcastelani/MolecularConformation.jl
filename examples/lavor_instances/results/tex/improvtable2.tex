
	\begin{table}[H]
        \centering
        {\footnotesize
        
        \begin{tabular}{||r |S[table-format=1.2e+2] |S[table-format=1.2e+2] |S[table-format=1.2e+2] |S[table-format=1.2e+2] |S[table-format=1.2e+2] |S[table-format=1.2e+2] | S[table-format=1.2e+2]||}
                \hline
				        $n_d$ & \multicolumn{1}{c|}{400} & \multicolumn{1}{c|}{500} & \multicolumn{1}{c|}{600} & \multicolumn{1}{c|}{700} & \multicolumn{1}{c|}{800} & \multicolumn{1}{c|}{900} & \multicolumn{1}{c||}{1000} \\
        \hline
        Quater. & 2.34e-03 & 2.45e-03 & 2.54e-03 & 2.61e-03 & 2.69e-03 & 2.72e-03 & 2.69e-03 \\
        Matrix & 2.43e-03 & 2.54e-03 & 2.63e-03 & 2.70e-03 & 2.80e-03 & 2.82e-03 & 2.79e-03 \\
        $t_q\slash t_m$ & \multicolumn{1}{c|}{0.96} & \multicolumn{1}{c|}{0.96} & \multicolumn{1}{c|}{0.96} & \multicolumn{1}{c|}{0.97} & \multicolumn{1}{c|}{0.96} & \multicolumn{1}{c|}{0.96} & \multicolumn{1}{c||}{0.96} \\
        \hline
	\end{tabular}}
	\caption{Pratical improvement ratio of QuaternionBP in relation to MatrixBP considering the geometric means of benchmark times over all instances with fixed $n_d$.}
	\label{table:improvlavor}
\end{table}